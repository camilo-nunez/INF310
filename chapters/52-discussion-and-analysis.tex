\chapter{Discussion and Analysis}

\section{Metric Evaluation Analysis} \label{sec:40:metrics}
Upon scrutiny of the aggregated data from the Table \ref{table:results:main}, it becomes evident that networks employing the proposed CABiFPN neck architecture demonstrated superior performance in both bounding box and mask predictions, when compared to those networks utilizing the conventional BiFPN neck. More specifically, the architecture that most proficiently predicted bounding boxes was a combination of EfficientNetV2-M~\cite{tan2021efficientnetv2} with CABiFPN. Conversely, the least effective architecture for this particular task was identified as InternImage-S~\cite{wang2023internimage} combined with BiFPN~\cite{tan2020efficientdet}. In terms of mask prediction, the ConvNeXt-S network~\cite{liu2022convnet} paired with CABiFPN produced the most favorable outcomes, while the least effective results were similarly attributed to the InternImage-S network~\cite{wang2023internimage} with neck BiFPN~\cite{tan2020efficientdet}.\\

It is noteworthy to address the implications of architectural choices on model complexity. The integration of a greater number of parameters in the proposed CABiFPN neck substantially elevates the model's depth-related complexity as compared to its BiFPN counterpart. This suggests that a more expansive neck architecture, such as CABiFPN, manifests a higher degree of computational sophistication than a less expansive structure like BiFPN. However, this elevated complexity is accompanied by more intricate internal representations within the neck architecture. Consequently, this demands additional training time and computational resources, aligning with the observations put forth in previous work~\cite{Bengio2009}.

\section{Explanation with Ablation CAM}
In an examination of the predictive capabilities of six computational models, variety of differentiated elements were noted with respect to object detection and mask generation in images with IDs \texttt{96825}, \texttt{171382}, and \texttt{509403}. These outcomes are elucidated in Figures \ref{fig:results:preds}, \ref{fig:results:preds:a1}, and \ref{fig:results:preds:a2}, showcasing key differences between models that implement CABiFPN and BiFPN as their respective neck architectures.\\

Focusing initially on the predictions related to image ID \texttt{509403}, as presented in Figure \ref{fig:results:preds}, it was observed that the models equipped with the CABiFPN neck architecture successfully identified all bounding boxes for the object instances present (comprising three instances of \texttt{'person'}, one of \texttt{'dog'}, and one of \texttt{'frisbee'}). Conversely, models utilizing the BiFPN neck were less proficient, particularly in identifying objects with smaller surface areas such as instances of \texttt{'frisbee'} and \texttt{'dog'}. It is worth noting that both sets of models (those with CABiFPN and those with BiFPN neck architectures) were able to accurately generate masks for the aforementioned instances.\\

Turning to the predictive outcomes for image ID \texttt{96825}, as illustrated in Figure \ref{fig:results:preds:a1}, all models uniformly succeeded in identifying the mask and the associated bounding box for the \texttt{'person'} instance that occupied the largest area. However, the models collectively failed to detect smaller object instances, notably those labeled as \texttt{'backpack'} and \texttt{'skis'}. Additional scrutiny revealed that models incorporating InternImage-S + CABiFPN and ConvNeXt-S + CABiFPN architectures produced an increased number of segmentation artifacts.\\

Lastly, concerning the two most frequently occurring instances in image ID \texttt{171382} (delineated in Figure \ref{fig:results:preds:a2}) none of the models were able to precisely pinpoint the bounding box for the most prevalent instance, labeled as \texttt{'traffic light'}. Nevertheless, these models demonstrated a higher success rate in generating corresponding masks. With regard to the second most frequently instance, the \texttt{'person'} instances, models employing the CABiFPN neck architecture predominantly excelled in identifying both the bounding boxes and associated masks. Recurring issues were observed with the ConvNeXt-S + CABiFPN architecture, which continued to generate artifacts in segmentation tasks.\\

While visualizations of the predictions may initially clarify the behavior of the models, it is necessary to delve deeper into the visual explanation provided by the Ablation CAM method.

\subsection{Ablation CAM visual explanation Image \texttt{509403}} \label{subsec:40:cam:509403}
Figures \ref{fig:results:cam:509403:1} and \ref{fig:results:cam:509403:2} present visualizations generated via Ablation CAM for models equipped with InternImage-S backbones and BiFPN and CABiFPN necks, respectively. A salient observation, corroborated by previous accuracy measurements, is the exclusive visualization of \texttt{'person'} instances by the model incorporating the BiFPN neck. Conversely, the model featuring the CABiFPN neck demonstrates the capability to identify multiple instances across classes.
Specific attention is drawn to layer 1 of Figure \ref{fig:results:cam:509403:1}, which belongs to the BiFPN neck model. Here, discriminative regions (marked in red) appear to be sporadically distributed across the instance, most prominently in the larger \texttt{'person'} instances. Such a pattern suggests that the BiFPN neck initially fails to leverage the substantial contributions that context features can offer. This limitation, however, appears to be ameliorated beginning with layer 2. In contrast, the CABiFPN neck model evidences the effective utilization of context features from layer 1 itself, accentuating elements derived from local context features, such as the limbs of \texttt{'person'} instances or smaller objects.\\

In the context of the ConvNeXt-S~+~BiFPN and ConvNeXt-S~+~CABiFPN models, Figures \ref{fig:results:cam:509403:3} and \ref{fig:results:cam:509403:4} reveal that both networks initially exhibit a low scattered distribution of smaller sub-regions. This is particularly noticeable in the network employing the BiFPN neck. This low dispersion implies that the fusion of context features transpires earlier in models equipped with a ConvNeXt-S backbone as compared to those with an InternImage-S backbone. Nevertheless, the CABiFPN neck model continues to outperform in highlighting discriminative regions imbued with high-level context features, such as the limbs of \texttt{'person'} instances or small objects like the \texttt{'frisbee'} instance.\\

Lastly, the EfficientNetV2-M~+~BiFPN and EfficientNetV2-M~+~CABiFPN models, visualized in Figures \ref{fig:results:cam:509403:5} and \ref{fig:results:cam:509403:6}, diverge from preceding findings. For the BiFPN neck model, the fusion of context features is apparent from layer 1, a feature not mirrored in the CABiFPN neck model where such fusion is discernible only in the innermost layers, ranging from layer 2 to layer 4. Notwithstanding these differences, both models converge towards highlighting similar discriminative sub-regions by Layer 5.

\subsection{Ablation CAM visual explanation Image \texttt{96825}}\label{subsec:40:cam:96825}
As mentioned in the begening of this section, image ID \texttt{96825} contains only three instances: \texttt{'person'}, \texttt{'backpack'}, and \texttt{'skis'}. The first being the instance with the largest area in the image, and the last two with much smaller areas compared to the first instance.

In this case, the visualizations of the models with the InternImage-S backbone and the BiFPN and CABiFPN necks are illustrated in Figures \ref{fig:results:cam:96825:1} and \ref{fig:results:cam:96825:2}, respectively. It can be observed and confirmed that the model with the BiFPN neck rapidly identifies the largest \texttt{'person'} instance in the first three layers without major issues. However, in the last two layers, it can be observed that both layer 4 and layer 5 start to generate dispersed regions in the instance detection, indicating that these latter layers generate more artifacts compared to the first three. In contrast, in Figure \ref{fig:results:cam:96825:2}, it can be seen that the model with the CABiFPN neck identifies the instance homogeneously, despite not detecting it in the first layer. Again, it is noticeable that in this model, the details of the \texttt{'person'} instances's limbs are better defined compared to the detections made by the model with the BiFPN neck.\\

Regarding the visualizations of the models with the ConvNeXt-S backbone, presented in Figures \ref{fig:results:cam:96825:3} and \ref{fig:results:cam:96825:4} for the BiFPN and CABiFPN necks, respectively, a similar behavior can be observed as described in Section \ref{subsec:40:cam:509403} for the same image, and as described above. A homogeneous distribution of regions of interest is observed across all five layers in both models with BiFPN and CABiFPN necks. The model with the CABiFPN neck achieves a more accurate identification of local context features of the instance's limbs in layer 2 compared to the model with the BiFPN neck.\\

A similar effect occurs in Figures \ref{fig:results:cam:96825:5} and \ref{fig:results:cam:96825:6} for the EfficientNetV2-M backbone, as all layers of both models with BiFPN and CABiFPN necks homogeneously identify the instance in their respective regions of interest, without showing significant representative differences in their visualization.

\subsection{Ablation CAM visual explanation Image \texttt{171382}}\label{subsec:40:cam:171382}
In the present study, Image ID \texttt{171382} serves as a particularly challenging example for evaluating the efficacy of instance detection across six pre-trained models. This complexity arises from a confluence of factors such as the sheer number of instances, variability in their dimensions and geometries, or the degree of overlapping among them.
Figures \ref{fig:results:cam:171382:1} to \ref{fig:results:cam:171382:6} provide visual representations to support our observations. Notably, it is exclusively the models incorporating the CABiFPN neck that demonstrate a superior ability in detecting a diverse array of instances, beyond the overrepresented \texttt{'person'} class.\\

In the specific of Figures \ref{fig:results:cam:171382:1} and \ref{fig:results:cam:171382:2}, models employing both InternImage-S backbone and BiFPN and CABiFPN necks primarily identify instances of the \texttt{'person'} class. Yet, the model featuring the CABiFPN neck exhibits the capability to discern \texttt{'person'} instances across a range of scales. In contrast, the model with the BiFPN neck only effectively identifies \texttt{'person'} instances when they are confined to a similar area and pose. This suggests that the BiFPN-based model is more prone to neglecting underrepresented contextual features within the image.\\

Further scrutiny of Figures \ref{fig:results:cam:171382:3} and \ref{fig:results:cam:171382:4} reveals nuanced behavior from the ConvNeXt-S~+~BiFPN and ConvNeXt-S~+~CABiFPN models. Again, the model with the CABiFPN neck outperforms its BiFPN counterpart in terms of instance detection. Interestingly, the CABiFPN model initiates with dispersed focal regions in its initial layers, particularly noticeable in the \texttt{'person'} instance depicted performing a jumping gesture over a \texttt{'skateboard'} instance, unlike other \texttt{'person'} instances where the gestures are more homogenous.\\

Lastly, an examination of Figures \ref{fig:results:cam:171382:5} and \ref{fig:results:cam:171382:6}, which correspond to the EfficientNetV2-M~+~BiFPN and EfficientNetV2-M~+~CABiFPN models, respectively, corroborates the aforementioned pattern. Here again, the CABiFPN model identifies a greater number of instances. However, it is only in the model's advanced layers that the fuse of high-level context features becomes more manifest, thereby accentuating key discriminative regions such as the limbs of the \texttt{'person'} instances.

\section{Validation of Visual Explanations through Imputation Techniques} \label{sec:validation-imputation}
As elaborated upon in Sections \ref{subsubsec:road} and \ref{sec:51:road-res}, a robust visual explanation is expected to result in an escalation of output confidence. To assess this, Tables \ref{tab:results:509403:comb-road}, \ref{tab:results:96825:comb-road}, and \ref{tab:results:171382:comb-road} present the variations in confidence scores for image IDs \texttt{509403}, \texttt{96825}, and \texttt{171382}, respectively. The employed strategy for this assessment is the Removal and Ordering Aware Decomposition (\textit{ROAD}), in conjunction with Noisy Linear Imputation as the imputation operator \(\mathcal{I}_l\) for the function \(\mathcal{I}_l(\boldsymbol{M}, \boldsymbol{x}_l)\). Here, \(\boldsymbol{M}\) represents the binary masks derived from the Ablation CAM visual explanations. The pixel elimination sequences applied are Least Relevant First (\textbf{LeRF}) and Most Relevant First (\textbf{MoRF}), as detailed in Section \ref{subsubsec:road}. Instances of these removal orders are illustrated in Figures \ref{fig:results:road:lrf} and \ref{fig:results:road:mrf}.\\

It is crucial to highlight that a higher confidence score does not inherently imply superior model performance. An elevated confidence score solely corroborates that the obtained visual explanation's mask is in alignment with the pixel modification strategy and, by extension, adapts well to the focal visual fields under investigation across different layers.\\

For image ID \texttt{509403}, Table \ref{tab:results:509403:comb-road} corroborates that the initial layers offer more precise visual explanations in the InternImage-S~+~BiFPN model, thus substantiating the observations made in Section \ref{subsec:40:cam:509403}. In contrast, the InternImage-S~+~CABiFPN model demonstrates stable confidence values across all layers. Comparable stability in confidence values is also noted for both the ConvNeXt-S~+~BiFPN and ConvNeXt-S~+~CABiFPN models. For the EfficientNetV2-M~+~BiFPN and EfficientNetV2-M~+~CABiFPN models, it is unequivocally observed that the final layers produce the highest confidence scores. This observation reiterates the layer fusion occurrences in the ConvNeXt-S~+~CABiFPN model and further substantiates that the terminal layer contributes most substantially to the visual explanation in both BiFPN and CABiFPN neck configurations.\\

Concerning image ID \texttt{96825}, as per Table \ref{tab:results:96825:comb-road}, the confidence value distribution across all models is largely homogenous, with a notable exception in the final two layers of the InternImage-S~+~BiFPN model. The latter possibly encompasses elements missed by the visual explanations, such as particular artifacts.\\

Finally, in the context of image ID \texttt{171382}, Table \ref{tab:results:171382:comb-road} exhibits trends akin to those observed for the InternImage-S~+~BiFPN model in Table \ref{tab:results:509403:comb-road}. Specifically, the most compelling confidence scores are registered in the initial layers, particularly within the first three layers of the network's neck architecture. This sharply contrasts with the elevated confidence values observed for the concluding layers in the InternImage-S~+~CABiFPN model, specifically layers 4 and 5. For the ConvNeXt-S~+~BiFPN model, the intermediate layers, specifically layers 2 to 4, yield the most refined visual explanations. Concurrently, the ConvNeXt-S~+~CABiFPN model records the apex of its confidence values in the final layer. In both EfficientNetV2-M~+~BiFPN and EfficientNetV2-M~+~CABiFPN configurations, it is discernible that the intermediate layers (layer 2 for the BiFPN model and layer 3 for the CABiFPN model) garner the most substantial confidence scores.